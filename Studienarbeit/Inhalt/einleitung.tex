\chapter{Einleitung}

\section{Ziel der Arbeit}
Diese Arbeit wurde im Rahmen einer Studienarbeit an der Dualen Hochschule Baden-Württemberg unter der Leitung von Prof. Dr. Karl Stroetmann angefertigt. Die Arbeit dient zur Unterstützung und Erweiterung der von Herrn Stroetmann gehaltenen Vorlesung \glqq Artificial-Intelligence\grqq \cite{stroetmann:2017}. Ziel der Arbeit ist es, die Vorlesung um ein praktisches Beispiel für Neuronale Netze zu erweitern. In den Vorlesungen von Herrn Stroetmann wird zur Veranschaulichung von Algorithmen und Methoden die, an mathematische Formulierungen angelehnte, Programmiersprache SetlX verwendet \cite{setlx:2017}. In dieser Programmiersprache sollte auch das Neuronale Netzwerk programmiert werden. \\
Als Basis des in dieser Studienarbeit implementierten Netzwerkes dient die Python Implementierung einer Zeichenerkennung von Michael Nielsen \cite{nielson:2017}.

\section{Aufgabe des neuronalen Netzwerks}
Ziel des in dieser Arbeit implementierten neuronalen Netzwerks ist es, handgeschriebene Zeichen zu erkennen und auszuwerten. Die eingelesenen Zeichen bestehen aus 28x28 Pixeln, welche in verschiedenen Graustufen dargestellt werden. Die Ziffern bestehen aus Werten zwischen 0 und 9. 
\begin{figure}[hbt]
	\centering
	\includegraphics[scale=0.6]{Bilder/handdrawn_digit}
	\caption{Handgeschriebene Ziffer 5 \cite{nielson:2017}} 
	\label{fig:handwritten_digit_5} 
\end{figure}
Abb. \ref{fig:handwritten_digit_5} zeigt ein Beispiel einer solchen Ziffer. Mit Hilfe des menschlichen Auges und Gehirns ist es für die meisten Menschen ohne Probleme möglich, zu erkennen, dass es sich hierbei um eine Ziffer mit dem Wert \glqq 5\grqq handelt. Eine Erkennung mittels herkömmlicher Computeralgorithmen hingegen stellt sich allerdings als sehr komplex und schwierig heraus. Gründe hierfür sind, dass beispielsweise verschiedene Ziffern durch unterschiedliche Handschriften signifikante Unterschiede aufweisen. Auch können beim Schreibvorgang einzelne Linien durch den Druck des Stiftes schwächer oder gar nicht abgebildet werden, was die gezeichnete Zahl ebenso variieren lässt. Diese und viele weitere Faktoren führen dazu, dass eine solche Zeichenerkennung mit Hilfe von einfachen Auswertealgorithmen zu hohen Fehlerraten führt und sehr aufwendig zu implementieren ist. \\ \\
Mit Hilfe eines neuronalen Netzwerks ist es bei solch einem Problem möglich, das Netzwerk automatisch mit Hilfe von Traningsdaten zu trainieren. Das bedeutet, dem Netzwerk wird eine möglichst große Menge an Testdaten übergeben und das Netzwerk lernt automatisch mit Hilfe dieser Daten, um das Problem mit möglichst geringer Fehlerrate zu lösen. Um dies bewerkstelligen zu können, müssen die Trainingsdaten aus folgenden Komponenten bestehen:
\begin{enumerate}
\item Eingabedaten (hier: Pixel des auszuwertenden Zeichens)
\item Erwartetes Ergebnis zu jeder Eingabe (hier: 5)
\end{enumerate}

\section{Verfügbarkeit des Programmcodes auf GitHub}
Der in dieser Studienarbeit entwickelte Programmcode, sowie sämtliche Dokumentation sind in GitHub unter folgender Adresse zu finden:
\\[0.2cm]
\hspace*{1.3cm}
\href{https://github.com/lucash94/Neural-Network-in-SetlX/}{https://github.com/lucash94/Neural-Network-in-SetlX}.
\\[0.2cm]
Im Verzeichnis \glqq Studienarbeit\grqq befindet sich diese Arbeit und das Verzeichnis \glqq setlx\grqq beinhaltet die eigentliche Implementierung in SetlX. Das dritte Verzeichnis \glqq res\grqq dient zur Aufbewahrung aller sonstigen Dateien und Aufzeichnungen der Studienarbeit.

\section{Aktuelle Relevanz von neuronalen Netzen}
Die Relevanz neuronaler Netzwerke nimmt im privaten Alltag immer mehr zu. Mittlerweile bieten große IT-Unternehmen Produkte für den Massengebrauch an, welche sich der Hilfe neuronaler Netzwerke bedienen. Einige populäre Beispiele des Einsatzes neuronaler Netzwerke sind:
\begin{enumerate}
\item Verbesserung der Übersetzungsergebnisse des Google Translaters wurden mittels neuronalen Netzen und einer hohen Anzahl an Trainingsdaten ermöglicht. Am 26.09.2016 wurde das Google Neural Machine Translation system (GNMT) in das Online-Tool eingeführt. \cite{gnmt:2017}
\item Das Programm AlphaGo des Unternehmens Google DeepMind ist spezialisiert auf das aus China stammende Brettspiel Go. Mit Hilfe eines neuronalen Netzes war AlphaGo das erste Computerprogramm, welches einen professionellen Go-Spieler schlagen konnte. \cite{alphago:2017}
\item Die Foto- und Videobearbeitungsapplikation Prisma nutzt ein neuronales Netzwerk um Fotos und Videos von Nutzern mit Effekten und Filtern basierend auf berühmten Kunstwerken zu versehen. \cite{prismaai:2017}
\end{enumerate}

\section{Aufbau der Arbeit}
Diese Arbeit ist in drei Kategorien unterteilt. Zu Beginn der Arbeit wird ein Überblick über das theoretische Wissen sowie den allgemeinen Aufbau und die Funktion neuronaler Netze gegeben. Anschließend wird die konkrete Umsetzung des Projektes in SetlX erläutert. Hierbei wird kurz die Beschaffung der Datensätze, gefolgt von der Hauptimplementierung, besprochen. Ebenso gibt es einen Abschnitt über ein weiteres SetlX-Programm, welches die Ausgabe des neuronalen Netzwerkes grafisch darstellt. \\
Der letzte Abschnitt der Arbeit befasst sich mit der Auswertung des Ergebnisses. Hierbei wird die Performance des finalen Programmes diskutiert sowie ein Fazit über den Erfolg oder Misserfolg der Arbeit gezogen.