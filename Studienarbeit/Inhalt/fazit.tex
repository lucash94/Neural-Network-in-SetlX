\chapter{Fazit und Ausblick}

\section{Auswertung des Ergebnisses}
Die Umsetzung der Handschriftenerkennung von Ziffern mittels einem neuronalen Feedforward-Netz wurde erfolgreich in SetlX implementiert. Die Erkennungsrate des Netzwerkes liegt ungefähr zwischen 94 und 96 Prozent. Die nachfolgende Ausgabe entspricht einem Durchlauf der $\mathtt{start.stlx}$-Datei. Hierbei wurden 10.000 Testdaten und 60.000 Trainingsdaten der MNIST-Datensätze verwendet. Die Epochenanzahl beträgt 30.

\begin{Verbatim}[ frame         = lines, 
                  framesep      = 0.3cm, 
                  firstnumber   = 1,
                  labelposition = bottomline,
                  numbers       = left,
                  numbersep     = -0.2cm,
                  xleftmargin   = 0.8cm,
                  xrightmargin  = 0.8cm,
                ]
    D:\DHBW\Neural-Network-in-SetlX\setlx>setlx start.stlx
    Reading file:   mnist_test.csv
    Image 10000 of 10000 imported
    End reading:    mnist_test.csv
    Reading file:   mnist_train.csv
    Image 10000 of 60000 imported
    Image 20000 of 60000 imported
    Image 30000 of 60000 imported
    Image 40000 of 60000 imported
    Image 50000 of 60000 imported
    Image 60000 of 60000 imported
    End reading:    mnist_train.csv
    Create Network
    Init Network
    Start SGD
    Epoch 1: 9427 / 10000
    Epoch 2: 9446 / 10000
    Epoch 3: 9534 / 10000
    Epoch 4: 9489 / 10000
    Epoch 5: 9581 / 10000
    Epoch 6: 9548 / 10000
    Epoch 7: 9559 / 10000
    Epoch 8: 9602 / 10000
    Epoch 9: 9543 / 10000
    Epoch 10: 9605 / 10000
    Epoch 11: 9571 / 10000
    Epoch 12: 9566 / 10000
    Epoch 13: 9603 / 10000
    Epoch 14: 9603 / 10000
    Epoch 15: 9598 / 10000
    Epoch 16: 9609 / 10000
    Epoch 17: 9587 / 10000
    Epoch 18: 9604 / 10000
    Epoch 19: 9609 / 10000
    Epoch 20: 9616 / 10000
    Epoch 21: 9591 / 10000
    Epoch 22: 9597 / 10000
    Epoch 23: 9605 / 10000
    Epoch 24: 9612 / 10000
    Epoch 25: 9588 / 10000
    Epoch 26: 9600 / 10000
    Epoch 27: 9598 / 10000
    Epoch 28: 9605 / 10000
    Epoch 29: 9582 / 10000
    Epoch 30: 9629 / 10000
    Time needed:    1255425ms		
\end{Verbatim}

\section{Performance der SetlX Implementierung}
Wie im vorherigen Kapitel zu sehen ist, beträgt die Durchlaufzeit des Programmes mit allen Datensätzen und 30 Epochen (Messzeit beginnt ab Aufruf der SGD-Funktion des Netzwerkes) 1255425 Millisekunden. Im Schnitt wurden ca. 1350000 Millisekunden benötigt, was 22,5 Minuten entspricht. Alle Messungen wurden auf einem Computer mit folgenden Hardwarekomponenten durchgeführt (nur relevante Komponenten sind aufgezählt):
\begin{itemize}
	\item \underline{CPU:} Intel Core i7-4720HQ @ 2.60GHz
	\item \underline{Arbeitsspeicher:} 16GB RAM
	\item \underline{Grafikkarte:} NVIDIA GeForce GTX 960M
\end{itemize}
Auffällig bei der Analyse der Laufzeit ist, dass die Durchlaufzeit der SetlX-Implementierung weit über der der Python-Implementierung liegt. Python benötigte für den Durchlauf der 30 Epochen auf dem selben Computer ungefähr 210000 Millisekunden. Das sind im Durchschnitt 1140000 Millisekunden (19 Minuten) weniger als in der SetlX Implementierung. \\
Basierend auf einer Reihe Performance-Tests, die auf Grund der langen Durchlaufzeiten durchgeführt wurden, stellte sich heraus, dass die Matrizen-Multiplikation in SetlX wesentlich zeitintensiver ist als in der Numpy-Bibliothek von Python. Der Faktor hierbei beträgt circa 6,5. \\
Eine Vermutung, wieso die SetlX Matrizen-Multiplikation wesentlich langsamer ist, ist dass in Python die Berechnung auf die Grafikkarte des Computers ausgelagert wird. Da die Programmiersprache Java (welche die Grundlage von SetlX bildet) plattformunabhängigkeit verspricht, wäre es möglich, dass hier keine Auslagerung auf die Grafikkarte stattfindet. Allerdings sind das nur Vermutungen und müssten auf Ebene der JAMA-Bibliothek (wird von SetlX zur Matrizen-Multiplikation verwendet) in Java, sowie der Numpy-Bibliothek in Python überprüft werden. \\ \\
Die gesamte Auswertung sowie alle hierfür verwendeten Programme sind im Verzeichnis
\\[0.2cm]
\hspace*{1.3cm}
/setlx/testing/
\\[0.2cm]
im GitHub-Repository zu finden. Die Datei $\mathtt{setlx\_performance\_evaluation.pdf}$ bietet eine Übersicht und Beschreibungen zu den durchgeführten Tests.

\section{Ausblick}